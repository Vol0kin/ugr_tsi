\documentclass[11pt,a4paper]{article}
\usepackage[spanish,es-nodecimaldot]{babel}	% Utilizar español
\usepackage[utf8]{inputenc}					% Caracteres UTF-8
\usepackage{graphicx}						% Imagenes
\usepackage[hidelinks]{hyperref}			% Poner enlaces sin marcarlos en rojo
\usepackage{fancyhdr}						% Modificar encabezados y pies de pagina
\usepackage{float}							% Insertar figuras
\usepackage[textwidth=395pt]{geometry}		% Anchura de la pagina
\usepackage[nottoc]{tocbibind}				% Referencias (no incluir num pagina indice en Indice)
\usepackage{enumitem}						% Permitir enumerate con distintos simbolos
\usepackage[T1]{fontenc}					% Usar textsc en sections
\usepackage{amsmath}						% Símbolos matemáticos
\usepackage{algorithm}						% Environtment algorithm
\usepackage{algpseudocode}					% Pseudocodigo

\algnewcommand\algorithmicforeach{\textbf{for each}}
\algdef{S}[FOR]{ForEach}[1]{\algorithmicforeach\ #1\ \algorithmicdo}

% Comando para poner el nombre de la asignatura
\newcommand{\asignatura}{Técnicas de los Sistemas Inteligentes}

% Configuracion de encabezados y pies de pagina
\pagestyle{fancy}
\lhead{Vladislav Nikolov, Carlos Núñez}
\rhead{\asignatura{}}
\lfoot{Grado en Ingeniería Informática}
\cfoot{}
\rfoot{\thepage}
\renewcommand{\headrulewidth}{0.4pt}		% Linea cabeza de pagina
\renewcommand{\footrulewidth}{0.4pt}		% Linea pie de pagina

\begin{document}
\pagenumbering{gobble}

% Pagina de titulo
\begin{titlepage}

\begin{minipage}{\textwidth}

\centering

\includegraphics[scale=0.5]{img/ugr.png}\\

\textsc{\Large \asignatura{}\\[0.2cm]}
\textsc{GRADO EN INGENIERÍA INFORMÁTICA}\\[1cm]

\noindent\rule[-1ex]{\textwidth}{1pt}\\[1.5ex]
\textsc{{\Huge PRÁCTICA 1\\[0.5ex]}}
\textsc{{\Large Técnicas de Búsqueda\\}}
\noindent\rule[-1ex]{\textwidth}{2pt}\\[3.5ex]

\end{minipage}

\vspace{0.4cm}

\begin{minipage}{\textwidth}

\centering

\textbf{Autores}\\ {Vladislav Nikolov Vasilev}\\ {Carlos Núñez Molina}\\[2ex]
\textbf{Rama}\\ {Computación y Sistemas Inteligentes}\\[2ex]
\vspace{0.3cm}

\includegraphics[scale=0.3]{img/etsiit.jpeg}

\vspace{0.7cm}
\textsc{Escuela Técnica Superior de Ingenierías Informática y de Telecomunicación}\\
\vspace{1cm}
\textsc{Curso 2018-2019}
\end{minipage}
\end{titlepage}

\pagenumbering{arabic}
\tableofcontents
\thispagestyle{empty}				% No usar estilo en la pagina de indice

\newpage

\setlength{\parskip}{1em}

\section{\textsc{Descripción General de la Solución}}

El aspecto fundamental de la práctica es cómo elegir qué gemas coger y en qué orden. Hay en total 23
gemas por nivel, de las cuales solo se necesitan coger 9. Si se hacen los cálculos, hay
${23\choose 9} \cdot 9! = 296541907200$ combinaciones posibles. Este número es inabarcable para el
A*, sin importar la estrategia usada, por lo que no podemos usarlo para que resuelva el nivel desde
cero: hace falta simplificar el problema.

Para reducir el número de posibilidades se ha usado una estrategia de \textit{clustering}, técnica de
aprendizaje no supervisado. La heurística detrás de esto es la siguiente: si nos encontramos en un
\textit{cluster} (grupo) de gemas, al estar estas gemas todas juntas, generalmente será una buena idea
(un buen plan) coger todas las gemas del \textit{cluster} antes de irse a otro. Por tanto, hemos
transformado el problema de qué gemas coger y en qué orden al problema de qué \textit{clusters} de
gemas coger y en qué orden. Como el número de \textit{clusters} es mucho menor que el de gemas, este
problema sí que es abordable. Para generar los \textit{clusters} se ha usado un algoritmo llamado
\textit{DBSCAN}. Su funcionamiento (implementado) es el siguiente: se va iterando por todas las gemas
del nivel; si esa gema no pertenece a un \textit{cluster} y no hay otra gema de algún \textit{cluster}
cerca suya se crea un nuevo \textit{cluster} y se asigna a él; después se ve qué otras gemas sin
\textit{cluster} están cerca de ésta y se asignan al mismo \textit{cluster}. Una gema está cerca de
otra si su distancia Manhattan es menor o igual a un parámetro $\varepsilon$ del método. En la
práctica se ha usado $\varepsilon = 3$, que es el que genera mejores \textit{clusters}. Para elegir el
\textit{tour} (camino) a través de los \textit{clusters} se ha usado un simple algoritmo de
\textit{Branch}\&\textit{Bound}, que devuelve un camino a través de \textit{clusters} de forma que en
total se consigan el número de gemas necesarias para abandonar el nivel, siendo el camino elegido en
función de la distancia entre los \textit{clusters} y la ``dificultad'' de cada \textit{cluster} (el
número de rocas, muros y enemigos en el \textit{cluster} y cómo de alejadas están sus gemas).

De esta forma, esta es la estrategia fundamental usada en la resolución de la práctica: agrupar las
gemas en \textit{clusters} e ir yendo de un \textit{cluster} a otro hasta tener 9 gemas, en cuyo caso
se planifica para abandonar el nivel.

La integración del comportamiento reactivo y deliberativo, a grandes rasgos y en pseudocódigo, es la
siguiente:

\begin{algorithm}
\caption{Integración del comportamiento reactivo-deliberativo (I)}
\begin{algorithmic}[1]
\Procedure{act}{$ $}
\If{$primer\_turno$} \Comment{Esto se hace en el constructor}
	\State crearClusetrsYCircuitos()
	\State $cluster\_actual \gets 0$
	\State buscarPlan($cluster\_actual$)
\algstore{bkact}
\end{algorithmic}
\end{algorithm}

\begin{algorithm}[H]
\caption{Integración del comportamiento reactivo-deliberativo (II)}
\begin{algorithmic}[1]
\algrestore{bkact}
\EndIf
\State \Comment{Plan creado cuando el jugador puede o va a morir en próx. turnos}
\If{$plan\_no\_morir$.isEmpty()}
	\State \textbf{return} $plan\_no\_morir$.first()
\EndIf
\If{$num\_gems \geq 9$} \Comment{Dirigirse a la salida si se puede salir}
	\State buscarPlanAbandonarNivel()
\EndIf
\If{$busqueda\_no\_terminada$} \Comment{Búsqueda puede tardar múlt. turnos}
	\State seguirBuscandoPlan()
\EndIf
\If{$busqueda\_terminada$ \textbf{and} $camino\_no\_encontrado$}
	\State $hay\_que\_replanificar \gets $ true
	\If{$num\_gems < 9$}
		\State removeCluster($cluster\_actual$) \Comment{Eliminar \textit{cluster} y recrear circuito}
		\State crearClusterYCircuito() \Comment{porque el \textit{cluster} es inaccesible}
	\EndIf
\EndIf
\If{$hay\_que\_replanificar$}
	\State buscarPlan()
\EndIf
\If{$busqueda\_terminada$}
	\If{$plan\_vacio$}
		\State $cluster\_actual \gets cluster\_actual + 1$
		\State buscarPlan($cluster\_actual$)
	\Else
		\State $accion \gets plan$.first()
	\EndIf
\EndIf
\State \Comment{Parte reactiva: ver si ejecutar la acción del plan o no}
\If{$enemigos\_cercanos$ \textbf{or} $muerte\_por\_roca$}
	\State crearPlanNoMorir()
	\State \textbf{return} $plan\_no\_morir$.first()
\EndIf
\If{$jugador\_choca\_con\_roca\_cayendo$}
	\State \textbf{return} $quedarse\_quieto$
\EndIf
\State \textbf{return} $accion$
\EndProcedure
\end{algorithmic}
\end{algorithm}

\section{\textsc{Comportamiento Reactivo}}

El comportamiento reactivo se ha centrado en decidir si ejecuto la acción del plan ``normal'' (el que es
obtenido usando el A* para coger las gemas o ir a la salida) o no. Las razones para no hacer esto son
dos: va a morir/puede morir en los siguientes turnos o el agente se va a chocar con una roca cayendo (con
lo que no va a poder ejecutar la acción del plan). En el caso de que pueda morir, se crea un plan
provisional ($plan\_no\_morir$) con las acciones que alejan al agente del peligro. Este plan se ejecutará
en vez del plan ``normal'' siempre que contenga acciones. Si va a chocarse contra una roca, simplemente
se queda quieto en ese turno y la acción a ejecutar la aplaza para ejecutarla el siguiente turno (si no
se repite esta situación).

También, en cada turno, se ve si hay que seguir con la búsqueda (en el caso de que todavía no haya
terminado el A*) o hay que buscar un nuevo plan (si ya tenemos gemas suficientes para abandonar el nivel,
si el plan actual está vacío o si no se ha encontrado camino). Esta parte no la incluiré en el
pseudocódigo porque ya la puse en el apartado anterior. A continuación se puede ver el comportamiento
reactivo en pseudocódigo:

\begin{algorithm}[H]
\caption{Pseudocódigo del comportamiento reactivo (I)}
\begin{algorithmic}[1]
\Procedure{Reactivo}{$ $}
\State $enemigos\_cercanos \gets $ false
\ForEach{$enemigo \in enemigos$}
	\State \Comment{Solo se tienen en cuenta los enemigos cercanos al jugador si no están}
	\State \Comment{``encerrados'' (ese enemigo puede llegar hasta el jugador)}
	\If{$enemigo\_cercano\_a\_jug$ \textbf{and} $camino\_enemigo\_conectado\_a\_jug$}
		\State $enemigos\_cercanos \gets$ true
	\EndIf
\EndFor
\If{$enemigos\_cercanos$}
	\State $hay\_que\_replanificar \gets$ true
	\State $casillas\_validas \gets \emptyset$
	\ForEach{$casilla \in caillas\_adyacentes\_jugador$}
		\If{$casilla$.tipo $\neq \lbrace \text{muro, roca} \rbrace$ \textbf{and}
		$casilla\_no\_roca\_encima$}
			\State $casillas\_validas$.add($casilla$)
		\EndIf
	\EndFor	
	\If{$casillas\_validas \neq \emptyset$}
		\State $casillas\_alejadas \gets casillas\_validas$.getCasillasAlejadasEnemigos()
		\If{$casillas\_alejadas \neq \emptyset$}
			\State $casilla\_elegia \gets casillas\_alejadas$.getClosestToGoal()
		\Else

\algstore{reactive}
\end{algorithmic}
\end{algorithm}

\begin{algorithm}[H]
\caption{Pseudocódigo del comportamiento reactivo (II)}
\begin{algorithmic}
\algrestore{reactive}
			\State $casilla\_elegida \gets casillas\_validas$.getFarthestFromEnemies()
		\EndIf
		
		\State $plan\_no\_morir$.add($acciones\_para\_llegar\_a\_casilla\_elegida$)
	\EndIf	
\EndIf
\If{$enemigos\_cercanos$}
	\State $acciones\_prediccion \gets plan\_no\_morir$.getFirstTwo()
\Else
	\State $acciones\_prediccion \gets plan\_normal$.getFirstTwo()
\EndIf
\If{$jugadorVaAMorirEnSiguientes2Turnos(acciones\_prediccion)$}
	\State $hay\_que\_replanificar \gets $ true
	\If{$jugadorTieneRocaArribaDerechaOIzquierda()$}
		\State $muerte\_por\_roca \gets$ true
	\EndIf
	\If{$muerte\_por\_roca$}
		\For{$casilla \in \lbrace centro, derecha, izquierda, abajo \rbrace$}
			\If{$jugador$.avanzar($acciones\_ir\_a\_casilla$).noHaMuerto()}
				\State $plan\_no\_morir$.add($acciones\_ir\_a\_casilla$)
			\EndIf
		\EndFor
	\EndIf
\EndIf
\If{$enemigos\_cercanos$ \textbf{or} $muerte\_por\_roca$}
	\State \textbf{return} $plan\_no\_morir$.getRemoveFirst()
\EndIf
\If{$accion \neq $ ACTION\_NIL}
	\State \Comment{Ver si no cambia posición y orientación tras ejecutar acción}
	\If{$jug\_sig\_estado = jugador$}
		\State \Comment{Ver que jugador no excave debajo de roca}
		\If{$jug\_no\_va\_a\_excavar\_debajo\_de\_roca$}
			\State \textbf{return} ACTION\_NIL
		\EndIf
	\EndIf
\EndIf
\State \textbf{return} $accion$ \Comment{Devolver acción del plan ``normal'' si se ha llegado aquí}
\EndProcedure
\end{algorithmic}
\end{algorithm}

\newpage

\section{\textsc{Comportamiento Deliberativo}}

Para la realización del comportamiento deliberativo se han implementado tres versiones del A*: una
que permite ir de una posición inicial a una final, una que permite ir de una posición inicial a
una final recogiendo las gemas de un \textit{cluster} y una parecida a la anterior pero sin posición
final. En las tres, aparte de las listas que utiliza el algoritmo, se ha añadido una lista de explorados
que contiene los nodos visitados y los expandidos para poder hacer una consulta rápida de qué nuevos
nodos expandir y cuáles no. La lista de nodos cerrados no se revisita, ya que la optimalidad no es lo más
importante en este caso al estar trabajando a nivel de \textit{cluster} y no de gemas. Como se valora
mucho la eficiencia en el tiempo, se ha modificado el A* para que las búsquedas se puedan ejecutar en
varios turnos, guardando la información internamente.

Se han usado 2 heurísticas distintas, una para el A* que busca un camino desde una casilla inicial a
otra final y otra para el A* que busca un camino que coja todas las gemas de un \textit{cluster}.

\begin{itemize}[label=\textbullet]
	\item \textbf{Heurística camino, \textit{getHeuristicDistance}}:
	Obtiene la longitud del camino (número de
	casillas de separación) que une ambas casillas, pudiendo atravesar rocas pero no muros. Si la
	casilla inicial y final difieren en su posición $x$ y su posición $y$, aumenta en 1 la distancia
	(ya que el agente tendrá que girar una vez como mínimo para llegar al destino). Debido a que
	puede atravesar las rocas (a diferencia del agente), esta es una heurística obtenida mediante un
	modelo relajado, con lo que es admisible y monótona.
	\item \textbf{Heurística gemas, \textit{getHeuristicGems}}:
	Crea un grafo donde las $n$ gemas dadas son 
	los $n$ nodos y escoge los $n-1$ lados más cortos de este grafo. El coste del lado entre la gema
	$a$ y $b$ se corresponde con el valor de la distancia entre $a$ y $b$, medida usando
	$getHeuristicDistance$.
	
	Esta heurística devuelve la suma de la distancia de la casilla inicial a la gema más cercana a
	esta, más la suma de los $n-1$ lados más cortos del grafo mencionado anteriormente, más la
	distancia de ir de la casilla final a su gema más cercana. La heurística sería admisible y
	monótona si se usara así, pero hemos decidido multiplicar esta suma por $\alpha = 2$, con lo que
	deja de ser admisible y monótona, a cambio de aproximar mejor la longitud real del camino, lo que
	hace que el A* tenga que explorar menos estados. Aunque no consiga la solución óptima de esta
	forma, hemos hecho pruebas y, de media, este camino no tiene más de 5 casillas de diferencia con
	el óptimo.
\end{itemize}

A continuación se procede a mostrar el pseudocódigo de la primera versión del A*, sobre la que se
comentarán brevemente las otras:

\begin{algorithm}[H]
\caption{Pseudocódigo del A* para ir de un inicio a un final sin lista de gemas}
\begin{algorithmic}[1]
\Procedure{buscarPlan}{$inicio, fin, casillas\_ignorar$}
\State $plan \gets \emptyset$
\State Inicializar variables y listas
\While{\textbf{not} $encontrado$ \textbf{and} $lista\_abiertos \neq \emptyset$ \textbf{and not}
		$timeout$}
	\State Comprobar si se ha producido $timeout$
	\State $nodo \gets lista\_abiertos$.getRemoveFirst()
	\If{$nodo$.posicion() $ = fin$}
		\State \textbf{$encontrado \gets $ true}
	\Else
		\State $vecinos \gets $ obtenerVecinos($nodo$.posicion())
		\ForEach{$vecino \in vecinos$}
			\If{posicionValida($vecino$) \textbf{and} $vecino \notin casillas\_ignorar$}
				\State $siguiente\_nodo \gets$ Información y costes
				\If{produceCaidaRoca($vecino$)}
					\State Simular caída de rocas y actualizar información
				\EndIf
				\If{ $vecino \notin lista\_explorados$}
					\State $lista\_abiertos$.addOrdenado($siguiente\_nodo$)
					\State $lista\_explorados$.add($siguiente\_nodo$)
				\EndIf{}
			\EndIf{}
		\EndFor
	\EndIf
	
	\State $lista\_cerrados$.addFirst($nodo$)
\EndWhile
\If{$timeout$}
	\State guardarInformacionBusquedaSinTerminar()
	\State \textbf{return} $plan$
\EndIf
\If{$encontrado$}
	\State $plan \gets$ procesarPlan($lista\_cerrados$.getFirst())
\EndIf
\State \textbf{return} $plan$
\EndProcedure
\end{algorithmic}
\end{algorithm}

Sobre esta implementación de la primera versión se tienen que realizar muy pocas modidicaciones para
llegar a las otras versiones. En el caso de la segunda, hay que pasarle una lista de gemas que coger,
comprobar en la línea 7 también si se han cogido todas las gemas y usar una u otra heurística al generar
un nuevo nodo. En el caso de la tercer versión, respecto a la anterior, no hay que pasarle una posición
final y en la línea 7 solo comprobar si se han cogido todas las gemas de la lista.

\end{document}

